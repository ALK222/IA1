\documentclass[twoside]{AiTeX}



\title{IA1}
\author{A.L.K.}
\date{septiembre 2020}
\begin{document}
%\datos{facultad}{universidad}{grado}{asignatura}{subtitulo}{autor}{curso}
\datos{Informática}{Universidad Complutense de Madrid}{Ingeniería informática}{Inteligencia Artificial 1}{Memoria de la sesión 1}{Alejandro Barrachina Argudo \\ Walid Bousnitra Bousnitra}{2021-2022}
\portadaApuntes
\pagestyle{empty}
\tableofcontents
\pagestyle{empty}
\justify

\chapterA{Parte 1: Introducción a la \gls{ia}}

\section{20Q}

Aunque ha tardado 22 preguntas en adivinar que estabamos pensando en una lámpara siempre se ha acercado mucho a la solución.
Las preguntas siempre estaban relacionadas con objetos electricos o muebles a partir de la pregunta 5.

\section{Irene de Renfe}

Tiene dos respuestas definidas para preguntas sobre como se encuentra. Parece detectar bastante bien preguntas sobre viajes. Si le preguntas cualquier cosa sobre compras te redirige a comprar billetes. Detecta faltas de ortografía y palabras incompletas.

\section{ELIZA}

Evade preguntas personales, a respuestas negativas se atasca en "¿Por qué no?", no parece entender lo que son preguntas sobre sentimientos y no reacciona bien a repeteiciones de preguntas.

\section{Rock Paper Scissors}

Si utilizas muchas veces la misma opción el algoritmo se acostumbra y empiezas a perder partidas. Si vas cambiando de opción de manera aleatoria la tasa de victorias es cercana al 50\%.

\section{Doodle guessing}

Excepto uno de los intentos ha conseguido adivinar todo lo dibujado, aun siendo dibujado a ratón sin demasiada técnica.

\section{Moral Machina}

% TODO : esto hay que explicarlo más

\chapterA{Parte 2: introducción a python}

Ejecicios entregados en el archivo ``Ejercicios.ipynb''.

\chapterA{Parte 3:}

\section{Sudoku}

Tras leer el código este parece más un algoritmo de fuerza bruta con poda para calcular las posibles resoluciones del problema más que un algoritmo de IA.


\printglossary[title={Glosario}]

\end{document}
